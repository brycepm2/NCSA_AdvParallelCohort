% Created 2023-10-23 Mon 21:32
% Intended LaTeX compiler: pdflatex
\documentclass[11pt]{article}
\usepackage[utf8]{inputenc}
\usepackage[T1]{fontenc}
\usepackage{graphicx}
\usepackage{longtable}
\usepackage{wrapfig}
\usepackage{rotating}
\usepackage[normalem]{ulem}
\usepackage{amsmath}
\usepackage{amssymb}
\usepackage{capt-of}
\usepackage{hyperref}
\usepackage[margin=0.5in]{geometry}
\author{Bryce Mazurowski}
\date{\today}
\title{Intro to OpenMP}
\hypersetup{
 pdfauthor={Bryce Mazurowski},
 pdftitle={Intro to OpenMP},
 pdfkeywords={},
 pdfsubject={},
 pdfcreator={Emacs 29.1 (Org mode 9.6.6)}, 
 pdflang={English}}
\begin{document}

\maketitle
NCSA Advanced Parallel Computing Cohort
Fall 2023

Due date: October 25, 2023

\section{What is loop scheduling? Describe at least two different schedules and discuss what type of workloads can benefit from each.}
\label{sec:org07bd50c}
Loop scheduling describes different strategies to partition a given
dataset used in a data parallelization framework. There are three main
loop scheduling strategies: static, dynamic, and guided.
\begin{itemize}
\item Static: full dataset is cut into equal chunks based on the number of
threads available to OpenMP. This is the default behavior and good
for situations where workload is relatively constant for any index
\item Dynamic: dataset is cut into chunks of size \(R\), in principle this
should lead to more chunks than there are threads available. Each
thread is assigned a chunk, and when it finishes the task on a given
chunk it a new one is assigned to it. This strategy can be useful
when the workload of a task depends on the index of the loop.
\item Guided: dataset is cut into equal portions and tasks are
started. Once a thread finishes, the remaining data is redistributed
equally again among all threads. This continues until the
redistribution pieces are a minimum size \(R\). This strategy is
useful for cases where the workload increases with index.
\end{itemize}

\section{What is a reduction clause? Why are they useful? Give a simple example of a situation where using a reduction clause would be appropriate.}
\label{sec:org8f0b93d}
Reduction clauses are instructions on how to combine data
members in a parallel section. They give each thread a private copy of
a variable that it can operate and and tell the program how each
thread's pieces fit together in the end. These greatly simplify
implementation and can avoid omp barrier that will slow down code.

An example where a reduction clause is appropriate would be computing
the maximum element of a large vector. Using a reduction clause on a
private variable max would allow each chunk to compute its own max and
then perform a final comparison once all threads are complete to get
one final value out.

\section{Write/find a simple serial program with a race condition or a nested loop (ask ChatGPT if you can’t find one).}
\label{sec:orge5fb4c7}
\begin{enumerate}
\item Parallelize the loop using OpenMP directives without explicitly declaring data contexts
\item Check to see if your code is doing what it was supposed to do (it should not if you do have a race condition)
\item Use data contexts (shared and private) to resolve the race condition and check to see if you get the expected results
\item Discuss your results
\end{enumerate}

This code has nested loops and race conditions. The code loops over a
vector of ints. In each iteration of the top loop, a second loop
occurs over a set number of points. In the second loop, a value
is assigned to a dataSet for each iteration. On inner loop exit, the data
structure is inserted into a std::map as a pair of outer loop int and
data structure. This is to mimic a specific problem I have faced in
numerical integration of nonlinear solid mechanics problems.

The nested loop requires a private variable or a race condition will
occur.

The map insert will intermittently crash the code in parallel because
one thread will try to insert in the map when another is resorting it
on insert. This is a tricky bug that a reduction clause should fix.

Insert your serial code here
\begin{verbatim}
// Bryce Mazurowski (2023)
// brycepm2@gmail.com

#include <iostream>
#include <vector>
#include <map>
#include <math.h>

using std::cout;
using std::endl;
using std::map;
using std::vector;

struct stateVar {
  // default constructor
  stateVar()=default;
  // default destructor
  ~stateVar()=default;
  // default copy constructor
  stateVar(stateVar& svIn)=default;

  // vector to store info for each intPt
  std::vector<double> vals;
};


// how to print data structure
std::ostream& operator<<(std::ostream& os, const stateVar& svObj) {
  cout << "stateVar: " << endl;
  std::vector<double> valsOut = svObj.vals;
  std::vector<double>::iterator iter_val = valsOut.begin();
  for (; iter_val != valsOut.end(); ++iter_val) {
    cout << (*iter_val) << ' ';
  }
  return os;
}

// map for each element
typedef map<int, stateVar*> stateVMap;

/** @brief loop over integration points insert number
 * into data structure for each int point
 */
void integrateLoop(stateVar* p_elemStateVar);

int main() {
  // set loop limits
  const int loopEnd = 1 << 4;

  // instantiate stateVarMap
  stateVMap svMap;
  for (int iElem = 0; iElem < loopEnd; ++iElem) {
    // create stateVar data structure for each elem
    stateVar* p_elemStateVar = new stateVar();
    // run integration loop
    integrateLoop(p_elemStateVar);
    // insert data structure into map
    svMap.insert({iElem, p_elemStateVar});
  }

  // loop over map and print each point
  // ALSO delete pointers within
  stateVMap::iterator iter_svMap = svMap.begin();
  for (; iter_svMap != svMap.end(); ++iter_svMap) {
    cout << "Key: " << iter_svMap->first << endl;
    cout << "struct: " << (*iter_svMap->second) << endl;
    delete iter_svMap->second;
  }
}


void integrateLoop(stateVar* p_elemStateVar) {
  // number of integration points
  const int nPts = 8;
  for (int iPt = 0; iPt < nPts; ++iPt) {
    double x = double(iPt);
    // make value some function
    const double value = std::pow(x,3.0) + 5*x*x + 13;
    // add value to data structure
    p_elemStateVar->vals.push_back(value);
  }
} 
\end{verbatim}

Insert your parallelized code without explicit data contexts here
\begin{verbatim}
// Bryce Mazurowski (2023)
// brycepm2@gmail.com

#include <iostream>
#include <vector>
#include <map>
#include <math.h>

#include <omp.h>

using std::cout;
using std::endl;
using std::map;
using std::vector;

struct stateVar {
  // default constructor
  stateVar()=default;
  // default destructor
  ~stateVar()=default;
  // default copy constructor
  stateVar(stateVar& svIn)=default;

  // vector to store info for each intPt
  std::vector<double> vals;
};


// how to print data structure
std::ostream& operator<<(std::ostream& os, const stateVar& svObj) {
  cout << "stateVar: " << endl;
  std::vector<double> valsOut = svObj.vals;
  std::vector<double>::iterator iter_val = valsOut.begin();
  for (; iter_val != valsOut.end(); ++iter_val) {
    cout << (*iter_val) << ' ';
  }
  return os;
}

// map for each element
typedef map<int, stateVar*> stateVMap;

/** @brief loop over integration points insert number
 * into data structure for each int point
 */
void integrateLoop(stateVar* p_elemStateVar);

int main() {
  // set loop limits
  const int loopEnd = 1 << 4;

  // instantiate stateVarMap
  stateVMap svMap;
  double sTime = omp_get_wtime();
  #pragma omp parallel
  {
    #pragma omp for
    for (int iElem = 0; iElem < loopEnd; ++iElem) {
      // create stateVar data structure for each elem
      stateVar* p_elemStateVar = new stateVar();
      // run integration loop
      integrateLoop(p_elemStateVar);
      // insert data structure into map
      svMap.insert({iElem, p_elemStateVar});
    }
  }
  double eTime = omp_get_wtime();

  cout << "Total time (s): " << (eTime - sTime) << endl;

  // loop over map and print each point
  stateVMap::iterator iter_svMap = svMap.begin();
  for (; iter_svMap != svMap.end(); ++iter_svMap) {
    cout << "Key: " << iter_svMap->first << endl;
    cout << "struct: " << (*iter_svMap->second) << endl;
    delete iter_svMap->second;
  }
  return 0;
}


void integrateLoop(stateVar* p_elemStateVar) {
  // number of integration points
  const int nPts = 18;
  for (int iPt = 0; iPt < nPts; ++iPt) {
    double x = double(iPt);
    // make value some function
    const double value = std::pow(x,3.0) + 5*x*x + 13;
    // add value to data structure
    p_elemStateVar->vals.push_back(value);
  }
} 
\end{verbatim}

Insert your final parallelized code here (race condition resolved)
\begin{verbatim}
 // Bryce Mazurowski (2023)
// brycepm2@gmail.com

#include <iostream>
#include <vector>
#include <map>
#include <math.h>

#include <omp.h>

using std::cout;
using std::endl;
using std::map;
using std::vector;

struct stateVar {
  // default constructor
  stateVar()=default;
  // default destructor
  ~stateVar()=default;
  // default copy constructor
  stateVar(stateVar& svIn)=default;

  // get size of vector for printing
  int getNumPts() { return vals.size(); }

  // check values within vector
  bool checkVals() {
    for (int iPt = 0; iPt < vals.size(); ++iPt) {
      double test = std::pow(iPt,3.0) + 5*iPt*iPt + 13;
      if (vals[iPt] != test) {
        return false;
      }
    }
    return true;
  }

  // vector to store info for each intPt
  std::vector<double> vals;
};


// how to print data structure
std::ostream& operator<<(std::ostream& os, const stateVar& svObj) {
  cout << "stateVar: " << endl;
  std::vector<double> valsOut = svObj.vals;
  std::vector<double>::iterator iter_val = valsOut.begin();
  for (; iter_val != valsOut.end(); ++iter_val) {
    cout << (*iter_val) << ' ';
  }
  return os;
}

// map for each element
typedef map<int, stateVar*> stateVMap;

/** @brief loop over integration points insert number
 * into data structure for each int point
 */
void integrateLoop(const int iElem, stateVar* p_elemStateVar);

/** @brief merge maps together for omp reduction
 */
void mergeMap(stateVMap& svMapOut, stateVMap& svMapIn) {
  svMapOut.merge(svMapIn);
}

// custom reduction for omp code
#pragma omp declare reduction(                          \
                              mergeMap :                \
                              stateVMap :                   \
                              mergeMap(omp_out, omp_in) \
                              )                         \
initializer (omp_priv=omp_orig)


int main() {
  // set loop limits
  const int loopEnd = 1 << 20;

  // instantiate stateVarMap
  stateVMap svMap;
  double sTime = omp_get_wtime();
#pragma omp parallel
  {
#pragma omp for reduction(mergeMap:svMap)
    for (int iElem = 0; iElem < loopEnd; ++iElem) {
      // create stateVar data structure for each elem
      stateVar* p_elemStateVar = new stateVar();
      // run integration loop
      integrateLoop(iElem, p_elemStateVar);
      // insert data structure into map
      svMap.insert({iElem, p_elemStateVar});
    }
  }
  double eTime = omp_get_wtime();

  cout << "Total time (s): " << (eTime - sTime) << endl;

  // loop over map and print each point
  // and delete created pointers
  stateVMap::iterator iter_svMap = svMap.begin();
  cout << "map length: " << svMap.size()
       << " correct is " << loopEnd << endl;
  for (; iter_svMap != svMap.end(); ++iter_svMap) {
    if (iter_svMap->second->getNumPts() != 18 &&
        iter_svMap->second->checkVals()) {
      cout << "Key: " << iter_svMap->first
           << " structLength: " << iter_svMap->second->getNumPts()
           << endl;
    } 
    delete iter_svMap->second;
  }

  return 0;
}


void integrateLoop(const int iElem, stateVar* elemStateVar) {
  // number of integration points
  const int nPts = 18;
  for (int iPt = 0; iPt < nPts; ++iPt) {
    double x = double(iPt);
    // make value some function
    const double value = std::pow(x,3.0) + 5*x*x + 13;
    // add value to data structure
    elemStateVar->vals.push_back(value);
  }
} 
\end{verbatim}

Discuss your results here

This code encounters 2 error categories:
\begin{enumerate}
\item If the nested iterator \texttt{iPt} is not declared private, the code can
skip points to add to the map. In practice this does not show
up, since the iterator is declared within the parallel region and
OpenMP automatically makes anything in there private.
\item If the map \texttt{svMap} is not private, each thread shares it. Sometimes
this results in a segmentation fault because the code is trying to
insert into the map while another may be reSorting it. Sometimes
this results in a element not being added to the map, so \texttt{map.size()}
ends up incorrect.
\end{enumerate}

To fix issue (2) I can't just make svMap private or it will always
return a map with zero size.
\begin{verbatim}
pragma omp for private(svMap)
Total time (s): 0.000262022
map length: 0 correct is 16
\end{verbatim}
I can do lastprivate, but that is still incorrect. That just takes the
value from the last thread to finish, not the complete thing.
\begin{verbatim}
pragma omp for lastprivate(svMap)
Total time (s): 0.000140905
map length: 2 correct is 16
\end{verbatim}
What I need is a reduction clause that will merge all of the maps.
\begin{verbatim}
pragma omp for reduction(mapMerge:svMap)
Total time (s): 0.0644159
map length: 32768 correct is 32768
\end{verbatim}

This was quite interesting. Implementing the reduction clause for the
map was a cool experience and this is a real problem I hit in
scientific computing. I had a better solution for that particular
problem, but was curious if OpenMP alone could do it without critical
sections. It seems that it indeed can.

I did not strictly use a data context, but reduction clauses imply
private variables for each thread. The reduction clause is just a
special case of a private variable.

\begin{center}
\begin{tabular}{rrrrrrr}
Threads & Run1 & Run2 & Run3 & Run4 & Avg & Speedup\\[0pt]
\hline
1 & 3.96231 & 3.97014 & 3.98751 & 3.96854 & 3.972125 & 1.\\[0pt]
2 & 2.33374 & 2.32085 & 2.31983 & 2.33406 & 2.32712 & 1.7068845\\[0pt]
4 & 1.50738 & 1.49919 & 1.49846 & 1.48885 & 1.49847 & 2.6507871\\[0pt]
\end{tabular}
\end{center}
Looking at the speedup, I do not get very good parallel improvement
for the case of 1048576 entries. This could be a results of bloat from
the maps. These things need to keep themselves sorted and I am now created
multiple entities and merge them.

Another option, which is what we
ended up doing in the research code, is to create a vector of pointers
to maps. The vector can be shared, because each thread will only
operate on one element. Then each thread creates its own map and
operates on it. This is likely a more efficient solution than the map
business within. 
\end{document}